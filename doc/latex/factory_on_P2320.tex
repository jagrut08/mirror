\documentclass[compress,table,xcolor=table]{beamer}
\input{beamer_mirror.tex}
\setlength{\paperwidth}{17.0cm}
\setlength{\paperheight}{10.0cm}
\setlength{\textwidth}{15.0cm}
\setlength{\textheight}{9.0cm}
\begin{document}
% ------------------------------------------------------------------------------
\title{Implementing Factory builder on top of P2320}
% - Overview -------------------------------------------------------------------
\section{Overview}
% ------------------------------------------------------------------------------
\begin{frame}
	\Large
	Implementing Factory builder on top of P2320
	\vfill
    \normalsize
	\begin{tabular}{|r|l|}
	\hline
	Document Number & P2356R0 \\
	\hline
	Date & 2021-04-09 \\
	\hline
	Reply-to & Matus Chochlik chochlik@gmail.com \\
	\hline
	Audience & SG7 \\
	\hline
	\end{tabular}
	\vfill
\end{frame}
% ------------------------------------------------------------------------------
\begin{frame}
    \Huge
    \centering{The goal}
    \vfill
    \LARGE
    \centering{(Semi-)automate the implementation\\of the\\Factory design pattern}
\end{frame}
% ------------------------------------------------------------------------------
\begin{frame}
\frametitle{What is meant by \say{Factory} here?}
  \Large
  \begin{itemize}
      \item Class that constructs instances of a \say{\larger Product} type.
      \item From some external representation:
      \begin{itemize}
        \large
      \item XML,
      \item JSON,
      \item YAML,
      \item a GUI,
      \item a scripting language,
      \item a relational database,
      \item \ldots
      \end{itemize}
  \end{itemize}
\end{frame}
% ------------------------------------------------------------------------------
\begin{frame}
  \frametitle{Factory builder}
    \Huge
  \begin{itemize}
    \item{Framework combining the following parts}
    \begin{itemize}
      \LARGE
    \item Meta-data
          \begin{itemize}
          \Large
        \item obtained from reflection.
          \end{itemize}
    \item Traits
          \begin{itemize}
          \Large
        \item units of code that handle various stages of construction,
        \item specific to a particular external representation,
        \item provided by user.
        \end{itemize}
      \end{itemize}
  \end{itemize}
\end{frame}
% ------------------------------------------------------------------------------
\begin{frame}
  \frametitle{Factory builder}
  \Huge
    \begin{itemize}
    \item{Used meta-data}
          \begin{itemize}
        \Large
        \item type name strings,
        \item list of meta-objects reflecting constructors,
        \item list of meta-objects reflecting constructor parameters,
        \item parameter type,
        \item parameter name,
        \item \ldots
          \end{itemize}
    \end{itemize}
\end{frame}
% ------------------------------------------------------------------------------
\begin{frame}
  \frametitle{Factory builder}
  \Huge
  \begin{itemize}
    \item{Units handling the stages of construction}
          \begin{itemize}
        \Large
        \item selecting the \say{best} constructor,
        \item invoking the selected constructor,
        \item conversion of parameter values from the external representation,
        \item may recursively use factories for composite parameter types.
        \end{itemize}
  \end{itemize}
\end{frame}
% ------------------------------------------------------------------------------
\begin{frame}[fragile]
  \frametitle{Used reflection features}
  \large
    \begin{itemize}
        \item \verb@^T@
        \item \verb@[: :]@\footnote{to get back the reflected type}
        \item \verb@meta::name_of@
        \item \verb@meta::type_of@
        \item \verb@meta::members_of@
        \item \verb@meta::is_constructor@
        \item \verb@meta::is_default_constructor@
        \item \verb@meta::is_move_constructor@
        \item \verb@meta::is_copy_constructor@
        \item \verb@meta::parameters_of@
        \item \verb@size(Range)@
        \item range iterators
    \end{itemize}
\end{frame}
% ------------------------------------------------------------------------------
\begin{frame}
  \frametitle{Reflection review}
    \Huge
    \begin{itemize}
    \item{The good\footnote{great actually!}}
        \begin{itemize}
        \Large
        \item{How extensive and powerful the API is}
        \end{itemize}
    \item{The bad}
        \begin{itemize}
        \Large
        \item{Didn't find much\footnote{some details follow}}
        \end{itemize}
    \item{The ugly}
        \begin{itemize}
        \Large
        \item{Some of the syntax\footnote{but then this is a matter of personal preference}}
        \end{itemize}
    \end{itemize}
\end{frame}
% ------------------------------------------------------------------------------
\begin{frame}[fragile]
\frametitle{What is missing}
    \Large
    \begin{itemize}
    \item The ability to easily unpack meta-objects from a range into a template,
        without un-reflecting them.
    \item I used the following workaround + \verb@make_index_sequence@:
    \end{itemize}
    \begin{lstlisting}[language=c++]
    template <typename Iterator>
    consteval auto advance(Iterator it, size_t n) {
      while(n-- > 0) {
        ++it;
      }
      return it;
    }
    \end{lstlisting}
    \begin{itemize}
    \item Details follow\ldots
    \end{itemize}
\end{frame}
% ------------------------------------------------------------------------------
\begin{frame}[fragile]
\frametitle{Metaobject range unpacking}
  \Large
  The goal is to unpack a metaobject range into a template like this:
    \begin{lstlisting}[language=c++]
    template <meta::info... MO>
    struct unpacked_range {
      constexpr static auto count = sizeof...(MO);
      // etc.
    };
    \end{lstlisting}
\end{frame}
% ------------------------------------------------------------------------------
\begin{frame}[fragile]
    \frametitle{Metaobject range unpacking (cont.)}
  \Large
  Unlike \verb@meta::info@ the \verb@detail::range@ type is not part of the
    public API, passing ranges as template parameters is not straightforward.
  \vfill
  So we are using this helper function, which makes the whole thing less generic.
    \begin{lstlisting}[language=c++]
    template <meta::info MO>
    consteval auto constructors_of() {
      return meta::members_of(MO, meta::is_constructor);
    }
    \end{lstlisting}
\end{frame}
% ------------------------------------------------------------------------------
\begin{frame}[fragile]
\frametitle{Metaobject range unpacking (cont.)}
    \large
    A helper:
    \begin{lstlisting}[language=c++]
    template <meta::info MO, size_t... I>
    consteval auto do_unpack_range(index_sequence<I...>) {
      return unpacked_range<*(*@\listinghl{advance}@*)(
        (*@\listinghl{constructors\_of}@*)<MO>().begin(),
        I)...>{};
    }
    \end{lstlisting}
    \vfill
    The unpack function:
    \begin{lstlisting}[language=c++]
    template <meta::info MO>
    consteval auto unpack_range() {
      return do_unpack_range<MO>(
		(*@\listinghl{make\_index\_sequence}@*)<
          size((*@\listinghl{constructors\_of}@*)<MO>())
        >{});
    }
    \end{lstlisting}
\end{frame}
% ------------------------------------------------------------------------------
\begin{frame}[fragile]
\frametitle{Metaobject range unpacking -- use case}
    \begin{lstlisting}[language=c++]
    template <meta::info>
    class my_base;

    template <typename Metaobjects>
    class my_derived;
    \end{lstlisting}

    \begin{lstlisting}[language=c++]
    template <meta::info ... MO>
    class my_derived<unpacked_range<MO...>>
     : public my_base<MO>... {
      // ...
    };
    \end{lstlisting}
\end{frame}
% ------------------------------------------------------------------------------
\begin{frame}[fragile]
\frametitle{Make ranges a \say{thing}}
    \Huge
    \begin{itemize}
        \item It would be great if the ranges were:
        \begin{itemize}
            \LARGE
            \item either \verb@meta::info@ themselves or
            \item had some public type like \verb@meta::range@
        \end{itemize}
    \end{itemize}
\end{frame}
% ------------------------------------------------------------------------------
\begin{frame}[fragile]
\frametitle{Make ranges a \say{thing} (cont.)}
    \Large
    Instead of:
    \begin{lstlisting}[language=c++]
    template <typename Range>
    class my_class;
    \end{lstlisting}
    either
    \begin{lstlisting}[language=c++]
    template <meta::info Range>
    class my_class;
    \end{lstlisting}
    or
    \begin{lstlisting}[language=c++]
    template <meta::range Range>
    class my_class;
    \end{lstlisting}
\end{frame}
% ------------------------------------------------------------------------------
\begin{frame}
    \LARGE
    \centering{But generally,}
    \vfill
    \Huge
    \centering{kudos to the implementers!}
    \vfill
    \large
    \centering{Some details on the factory builder follow\footnote{it there's interest}}
\end{frame}
% - Details --------------------------------------------------------------------
\section{Details}
% ------------------------------------------------------------------------------
\begin{frame}[fragile]
\frametitle{The mirror reflection utilities}
    \Large
    \begin{itemize}
        \item \url{https://github.com/matus-chochlik/mirror}
        \item implements the factory builder framework and some traits:
        \begin{itemize}
            \large
            \item simple input from \verb@iostreams@,
            \item input from JSON (using RapidJSON),
            \item input from a GUI (using Qt5/QML),
            \item others are planned.
        \end{itemize}
        \item plans for some additional use-cases:
        \begin{itemize}
            \large
            \item serialization/de-serialization,
            \item Python bindings,
            \item \ldots
        \end{itemize}
        \item There is an older implementation using manually-provided meta-data:
            \url{https://sourceforge.net/projects/mirror-lib/}
    \end{itemize}
\end{frame}
% ------------------------------------------------------------------------------
\begin{frame}[fragile]
\frametitle{Factory builder -- test classes}
    \begin{lstlisting}[language=c++]
    class point {
    public:
      point() noexcept = default;

      point(float v) noexcept
        : _x{v} , _y{v} , _z{v} {}

      point(float x, float y, float z) noexcept
        : _x{x} , _y{y} , _z{z} {}

      // ...
    private:
      float _x{0.F};
      float _y{0.F};
      float _z{0.F};
    };
    \end{lstlisting}
\end{frame}
% ------------------------------------------------------------------------------
\begin{frame}[fragile]
\frametitle{Factory builder -- test classes}
    \begin{lstlisting}[language=c++]
    class triangle {
    public:
      triangle() noexcept = default;

      triangle(point a, point b, point c)
        : _a{a}
        , _b{b}
        , _c{c} {}

      // ...
    private:
      point _a;
      point _b;
      point _c;
    };
    \end{lstlisting}
\end{frame}
% ------------------------------------------------------------------------------
\begin{frame}[fragile]
\frametitle{Factory builder -- test classes}
    \begin{lstlisting}[language=c++]
    class tetrahedron {
    public:
      tetrahedron() noexcept = default;
      tetrahedron(const triangle& base, const point& apex)
        : _base{base}
        , _apex{apex} {}

      // ...
    private:
      triangle _base;
      point _apex;
    };
    \end{lstlisting}
\end{frame}
% ------------------------------------------------------------------------------
\begin{frame}[fragile]
\frametitle{Factory builder -- JSON input}
    \begin{lstlisting}[basicstyle=\scriptsize\ttfamily]
    {
        "base": {
            "a": {
                "x": 2.0,
                "y": 0.0,
                "z": 0.0
            },
            "b": {
                "x": 0.0,
                "y": 1.0,
                "z": 0.0
            },
            "c": {
                "x": 0.0,
                "y": 0.0,
                "z": 1.0
            }
        },
        "apex": {
            "v": 0.0
        }
    }
    \end{lstlisting}
\end{frame}
% ------------------------------------------------------------------------------
\begin{frame}[fragile]
\frametitle{Factory builder -- JSON factory, usage}
    \large
    Working example\footnote{
        \url{https://github.com/matus-chochlik/mirror/blob/develop/example/factory/rapidjson.cpp}}:
    \begin{lstlisting}[language=c++]
    void print_info(const test::tetrahedron&);
    const auto json_str = ...;

    rapidjson::Document json_doc;
    const rapidjson::ParseResult parse_result{
        json_doc.Parse(json_str)};

    if(parse_result) {
        using namespace mirror;
        (*@\listinghl{factory\_builder}@*)<rapidjson_factory_traits> builder;
        auto factory = builder.(*@\listinghl{build}@*)<test::tetrahedron>();
        print_info(factory.(*@\listinghl{construct}@*)({json_doc}));

    }
    \end{lstlisting}
\end{frame}
% ------------------------------------------------------------------------------
\end{document}
